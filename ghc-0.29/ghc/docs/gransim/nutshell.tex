% Time-stamp: <Wed Jun 26 1996 22:59:47 Stardate: [-31]7719.58 hwloidl>
% 
% In a nutshell
% Shortest description of GranSim (part of the paper
%  ``Data-intensive Programs in Parallel Haskell'',
%  PW Trinder, K Hammond, HW Loidl, SL Peyton Jones, J Wu
%  in Glasgow FP Workshop 1996
%
% ----------------------------------------------------------------------------

% wicsbook style to be used for the final proceedings
%\documentstyle[12pt,a4,fp-workshop,psfig]{article}
\documentstyle[wicsbook,psfig]{article}

\begin{document}

\title{GranSim in a Nutshell}

\author{
 \begin{tabular}{c} 
   Hans Wolfgang Loidl   \\
   Dept.\ of Computing Science  \\
   University of Glasgow \\
   {\tt hwloidl@dcs.gla.ac.uk} 
 \end{tabular}
}

\date{}

\maketitle

GranSim is a   simulator for the  parallel  execution of  annotated Haskell
programs. 
% Parallelism  is introduced via the  {\tt `par`} annotation, which
% creates a `spark' (a potential parallel thread)  for the first argument and
% returns the  result of the second  argument. 
A slightly modified  code generator of ghc  instruments  the generated code
for checking data  locality  and  maintaining  a clock for  each  simulated
processor.  The simulation   itself is  event  driven.  Via  runtime-system
options GranSim  allows to simulate   a  wide range of  different  parallel
architectures, different  processors and  optionally special features  like
thread migration.

GranSim has  been developed  around the   same time as   GUM and  they  are
building  on  each others experiences.   Specifically, both systems use the
same method for synchronising  parallel threads and  to a large extend  the
same code for communicating data.  GranSim also allows to simulate features
not   existing in  GUM  like   different   methods  for  packing  data  and
synchronous, incremental communication as used in GRIP.

For  parallelising  big functional   programs the  distinction  between two
setups of GranSim  proved to be crucial:  The GranSim-Light setup simulates
an idealised  machine by assuming zero cost  communication  and an infinite
number of processors. This setup is typically used in an early stage of the
parallelisation to study the  total parallelism inherent  to the algorithm. 
The plain GranSim setup performs a very accurate modelling of communication
costs but limits the number of processors by the word size  of the machine. 
This is used to  simulate the execution  on a particular parallel machine.  
Furthermore, a  set of visualisation tools  allows to  show the activity of
the  whole machine, of  each processor and  of each thread.  Another set of
tools visualises the granularity of the generated threads.
 


\end{document}

